\documentclass[10pt]{article}  

%%%%%%%% PREÁMBULO %%%%%%%%%%%%
\title{Reporte de Laboratorio}
\usepackage[spanish]{babel} %Indica que escribiermos en español
\usepackage[utf8]{inputenc} %Indica qué codificación se está usando ISO-8859-1(latin1)  o utf8  
\usepackage{amsmath} % Comandos extras para matemáticas (cajas para ecuaciones,
% etc)
\usepackage{amssymb} % Simbolos matematicos (por lo tanto)
\usepackage{longtable} %agregadom para hacer tablas
\usepackage{xltabular}
\usepackage{graphicx} % Incluir imágenes en LaTeX
\usepackage{color} % Para colorear texto
\usepackage{subfigure} % subfiguras
\usepackage{float} %Podemos usar el especificador [H] en las figuras para que se
% queden donde queramos
\usepackage{capt-of} % Permite usar etiquetas fuera de elementos flotantes
% (etiquetas de figuras)
\usepackage{sidecap} % Para poner el texto de las imágenes al lado
	\sidecaptionvpos{figure}{c} % Para que el texto se alinie al centro vertical
\usepackage{caption} % Para poder quitar numeracion de figuras
\usepackage{commath} % funcionalidades extras para diferenciales, integrales,
% etc (\od, \dif, etc)
\usepackage{cancel} % para cancelar expresiones (\cancelto{0}{x})
 
\usepackage{anysize} 					% Para personalizar el ancho de  los márgenes
\marginsize{2cm}{2cm}{2cm}{2cm} % Izquierda, derecha, arriba, abajo

\usepackage{appendix}
\renewcommand{\appendixname}{Apéndices}
\renewcommand{\appendixtocname}{Apéndices}
\renewcommand{\appendixpagename}{Apéndices} 
% Para que las referencias sean hipervínculos a las figuras o ecuaciones y
% aparezcan en color
\usepackage[colorlinks=true,plainpages=true,citecolor=blue,linkcolor=blue]{hyperref}
%\usepackage{hyperref} 
% Para agregar encabezado y pie de página
\usepackage{fancyhdr} 
\pagestyle{fancy}
\fancyhf{}
\fancyhead[L]{\footnotesize UNI} %encabezado izquierda
\fancyhead[R]{\footnotesize Fac. de Ciencias}   % dereecha
\fancyfoot[R]{\footnotesize Reporte}  % Pie derecha
\fancyfoot[C]{\thepage}  % centro
\fancyfoot[L]{\footnotesize laboratorio de qu\'imica.}  %izquierda
\renewcommand{\footrulewidth}{0.4pt}


\usepackage{listings} % Para usar código fuente
\definecolor{dkgreen}{rgb}{0,0.6,0} % Definimos colores para usar en el código
\definecolor{gray}{rgb}{0.5,0.5,0.5} 
% configuración para el lenguaje que queramos utilizar
\lstset{language=Matlab,
   keywords={break,case,catch,continue,else,elseif,end,for,function,
      global,if,otherwise,persistent,return,switch,try,while},
   basicstyle=\ttfamily,
   keywordstyle=\color{blue},
   commentstyle=\color{red},
   stringstyle=\color{dkgreen},
   numbers=left,
   numberstyle=\tiny\color{gray},
   stepnumber=1,
   numbersep=10pt,
   backgroundcolor=\color{white},
   tabsize=4,
   showspaces=false,
   showstringspaces=false}

\newcommand{\sen}{\operatorname{\sen}}	% Definimos el comando \sen para el seno
%en español

\title{Seguridad y materiales de laboratorio}
% Basada en la plantilla para reportes UPIITA de  Overleaf

%%%%%%%% TERMINA PREÁMBULO %%%%%%%%%%%%

\begin{document}

\input{./inputs/caratula.tex}

\tableofcontents

\newpage

\section{Objetivos del Experimento}

\begin{itemize}
	\item Mediante procesos físicos separar una mezcla en sus componentes, analizando el porcentaje en peso de estas
	\item Analizar y comprender las etiquetas de los envases de los compuestos colocados en la campana extractora por el docente.
	\item Comprender y analizar la diferencia de temperatura de la llama en conbusti\'on e incompleta del mechero bunsen.
\end{itemize}

\section{Fundamento Teorico}
\subsection{Metodos de separacion Fisica de las mezclas}
Los métodos de separación de mezclas son técnicas físicas que se usan para separar los componentes de una mezcla, estas se basan en diferencias entre las propiedades físicas de los componentes de una mezcla, tales como: punto de ebullición, densidad, presión de vapor, punto de fusión, solubilidad, entre otros.

La importancia de estos métodos es que nos permiten obtener sustancias puras a partir de sus mezclas, sin alterar su estructura química.
ejemplos de métodos de separación física de las mezclas:
\begin{itemize}
	\item \textbf{Filtracion}: La filtración consiste en pasar una mezcla heterogénea líquido-sólido por un material filtrante que retiene el sólido y deja pasar el líquido. Los filtros pueden ser de diferentes materiales, cuya principal característica es ser poroso.
	\item \textbf{Vaporizacion}:  Es un método de separación que consiste en separar las sustancias disueltas en el líquido, esto se debe a que estas sustancias tienen diferentes puntos de ebullición, debido a esto el que tenga el punto de ebullición más baja se evapora separandola de la otra sustancia con un punto de ebullición más alto.
\end{itemize}

\section{Experimento 1: : Separación de una mezcla de Caliza, Sal común y Arena }
\subsection{Descripción del equipo y materiales utilizados}
\begin{itemize}
    \item Mechero bunsen
    \item Papel de filtro lento
    \item Embudo de 75 mm                                     
    \item Probeta graduada
    \item Bagueta de vidrio                                         
    \item Capsula de porcelana  
    \item Papel aluminio                                              
    \item Pipeta
    \item Frasco lavador (Pizeta)                                
    \item Luna o vidrio de reloj
    \item Mezcla de caliza, sal común y arena            
    \item Estufa 
\end{itemize}

\subsection{Procedimiento experimental}
\subsection{hoja de datos}
\subsection{Calculos y resultados}
\section{Bibliografía}
\begin{itemize}
	\item  Serway. F\'isica. Editorial McGraw-Hill (1992).\\
	\item Tipler. Física. Editorial Revert\'e (1994). \\
	\item Taylor, J.R. (2014) Introducci\'on al Análisis de errores - reverte, Taylor 2 Muestra.pdf. Disponible en: https://www.reverte.com/media/reverte/files/book-attachment-3746.pdf (Accessed: April 14, 2023).
\end{itemize}
\end{document}
